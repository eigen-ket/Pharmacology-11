\documentclass[10pt]{amsart}
\usepackage{lmodern}

\makeatletter
\ifcase \@ptsize \relax% 10pt
  \newcommand{\miniscule}{\@setfontsize\miniscule{4}{5}}% \tiny: 5/6
\or% 11pt
  \newcommand{\miniscule}{\@setfontsize\miniscule{5}{6}}% \tiny: 6/7
\or% 12pt
  \newcommand{\miniscule}{\@setfontsize\miniscule{5}{6}}% \tiny: 6/7
\fi
\makeatother

\usepackage{changepage}
%     If your article includes graphics, uncomment this command.
\usepackage{colortbl}
%\usepackage{graphicx}
\usepackage{leftidx}
\usepackage{mathtools}
\usepackage{graphicx}

\usepackage{expl3}
\ExplSyntaxOn
\cs_new_eq:NN \fpeval \fp_eval:n
\cs_new_eq:NN \clistitem \clist_item:Nn
\cs_new_eq:NN \foreachint \int_step_inline:nnnn
\ExplSyntaxOff

\usepackage{pgf}
\usepackage{pgf-spectra}
%\usepackage{fp}
%\usetikzlibrary{fpu}
%\usetikzlibrary{fpu}

%\pgfkeys{/pgf/fpu=true}

\newcommand{\eq}{=}
\newcommand{\setvalue}[1]{\pgfkeys{/variables/#1}}
\newcommand{\getvalue}[1]{\pgfkeysvalueof{/variables/#1}}
\newcommand{\declare}[1]{%
 \pgfkeys{
  /variables/#1.is family,
  /variables/#1.unknown/.style = {\pgfkeyscurrentpath/\pgfkeyscurrentname/.initial = ##1}
 }%
}
\pgfkeys{/pgf/number format/.cd ,precision=12,sci generic={exponent={\times 10^{#1}}}}
%\pgfset{fpu=true}


\usepackage{gnuplottex}
\usepackage{siunitx}
%\usepackage{subcaption}
%\usepackage{subfigure}
\usepackage{float}
\newcommand{\sinc}{\operatorname{sinc}}
\newcommand{\rect}{\operatorname{rect}}
\newcommand{\wll}{\textcolor{white}{123}}
\newcommand{\tot}{\text{tot}}
\newcommand{\ptl}{\text{ptl}}
\newcommand{\ove}{\operatorname{vec}}
\newcommand{\apr}{\text{apparatus}}
\newcommand{\dtr}{\text{dtr}}
\newcommand{\initial}{\text{initial}}
\newcommand{\final}{\text{final}}
\newcommand{\BS}{\operatorname{BS}}
\newcommand{\path}{\text{path}}
\newcommand{\up}{\uparrow}
\newcommand{\down}{\downarrow}
%\usepackage[dvipsnames]{xcolor}

%\usepackage[x11names]{xcolor}
\newtagform{blue}{\color{blue}(}{)}
%\usepackage[dvipsnames]{xcolor}

%\usepackage{pgfplots} 
 %   \usetikzlibrary{intersections}
    % use this `compat' level or higher so that TikZ coordinates don't have to be prefixed
    % with `axis cs:'
  %  \pgfplotsset{width=15cm,compat=1.11}
\usepackage{amsthm}
\usepackage{amsmath,amssymb}
\usepackage{helvet}
%\usepackage[leqno]{amsmath}
%\usepackage{blindtext}
\usepackage{bbm}

\makeatletter
\newcommand{\newparallel}{\mathrel{\mathpalette\new@parallel\relax}}
\newcommand{\new@parallel}[2]{%
  \begingroup
  \sbox\z@{$#1T$}% get the height of an uppercase letter
  \resizebox{!}{\ht\z@}{\raisebox{\depth}{$\m@th#1/\mkern-5mu/$}}%
  \endgroup
}
\makeatother



\DeclareSymbolFont{extraup}{U}{zavm}{m}{n}
\DeclareMathSymbol{\varheart}{\mathalpha}{extraup}{86}
\DeclareMathSymbol{\vardiamond}{\mathalpha}{extraup}{87}
\newcommand{\redheart}{\textcolor{red}{$\varheart$}}
\newcommand{\heart}{\ensuremath\varheart}

%\definecolor{orange}{rgb}{1.0, 0.7, 0}
\definecolor{awesome}{rgb}{1.0, 0.13, 0.32}
\definecolor{}{rgb}{0.0, 0.0, 0.60}
\makeatletter\newcommand{\leqnomode}{\tagsleft@true\let\veqno\@@leqno}
\newcommand{\reqnomode}{\tagsleft@false\let\veqno\@@eqno}\makeatother

\makeatletter
\newcommand{\pushright}[1]{\ifmeasuring@#1\else\omit\hfill$\displaystyle#1$\fi\ignorespaces}
\newcommand{\pushleft}[1]{\ifmeasuring@#1\else\omit$\displaystyle#1$\hfill\fi\ignorespaces}
\makeatother

\makeatletter
\newcommand{\specialcell}[1]{\ifmeasuring@#1\else\omit$\displaystyle#1$\ignorespaces\fi}
\makeatother

%\newcommand{\subsec}[1]{\begin{adjustwidth}{-0.1in}{0in}\Large\textbf{\textcolor{darkblue}{#1}}\end{adjustwidth}\normalsize \vspace{1ex}}

\newcommand{\linebreakc}{\textcolor{white}{bl}\\}
\usepackage[english]{babel}
\usepackage[utf8]{inputenc}
% use KoTeX package for Korean 
\usepackage{kotex}
% for the fancy \koTeX logo
\usepackage{kotex-logo}


\newcommand{\expp}[1]{\exp\left(#1\right)}
\newcommand{\expb}[1]{\exp\left[#1\right]}
\newcommand{\sinb}[1]{\sin\left[#1\right]}
\newcommand{\cosb}[1]{\cos\left[#1\right]}
\newcommand{\sinp}[1]{\sin\left(#1\right)}
\newcommand{\cosp}[1]{\cos\left(#1\right)}

\newcommand{\logb}[1]{\log\left[#1\right]}
\newcommand{\lnb}[1]{\ln\left[#1\right]}
\newcommand{\logp}[1]{\log\left(#1\right)}
\newcommand{\lnp}[1]{\ln\left(#1\right)}


\usepackage{hyperref}
\usepackage[nameinlink,capitalize]{cleveref}
\hypersetup{
    colorlinks=true,
    linkcolor=blue,
    filecolor=magenta,      
    urlcolor=blue,
    pdftitle={Homework 4 : Lindblad equation },
    bookmarks=true,
    pdfpagemode=FullScreen,
    }
\let\oldref\ref
\renewcommand{\ref}[1]{(\oldref{#1})} 
\def\equationautorefname~#1\null{(#1)\null}
\newcommand{\tagg}[1]{\stepcounter{equation}\tag{\theequation}\label{#1} }
\urlstyle{same}
\newcommand{\taggg}[1]{\tag{#1}\label{#1} }
%\usepackage{subfig}


%\numberwithin{equation}{section}

%\numberwithin{example}{section}
%\usepackage{subfig}

%\numberwithin{subfigure}{figure}

%\captionsetup[subfigure]{subrefformat=simple,labelformat=simple,listofformat=subsimple}
%\renewcommand\thesubfigure{(\alph{subfigure})}


\newtheorem{theorem}{Theorem}
\newtheorem*{theorem*}{Theorem}
\numberwithin{theorem}{section}

\newtheorem{lemma}{Lemma}
\newtheorem{proposition}{Proposition}

\newtheorem{exercise}{Exercise}
\newtheorem*{exercise*}{Exercise}
\crefname{exercise}{Exercise}{exercises}

\newtheorem{example}{Example}
\newtheorem*{example*}{Example}
\numberwithin{example}{section}
%\newtheorem{lemma}[lemma]{Lemma}
%\newtheorem{proposition}[theorem]{Proposition}
\newtheorem{corollary}[theorem]{Corollary}

%\newtheorem{exercise}[theorem]{Exercise}





\newtheorem{remark}[theorem]{Remark}
\newtheorem*{remark*}{Remark}
\newtheorem{problem}{Problem}
\newtheorem*{problem*}{Problem}

\newtheorem{innercustompro}{Problem}
\newenvironment{prob}[1]
  {\renewcommand\theinnercustompro{#1}\innercustompro}
  {\endinnercustompro}




\newtheorem{innercustomlem}{Lemma}
\newenvironment{lem}[1]
  {\renewcommand\theinnercustomlem{#1}\innercustomlem}
  {\endinnercustomlem}
  
\newtheorem{innercustomthm}{Theorem}
\newenvironment{thm}[1]
  {\renewcommand\theinnercustomthm{#1}\innercustomthm}
  {\endinnercustomthm}


\newenvironment{prf}
  {\begin{proof}[\textbf{\textcolor{magenta}{Proof}}\\]}
  {\end{proof}}
\makeatletter
  \renewcommand\@upn{\textit}
\makeatother



\newenvironment{solution}
  {\begin{proof}[\textbf{\textcolor{blue}{Solution}}\\]}
  {\end{proof}}
\makeatletter
  \renewcommand\@upn{\textit}
\makeatother

\newcommand{\lf}{\left}
\newcommand{\rg}{\right}
\theoremstyle{definition}
\newtheorem{definition}{Definition}
\newtheorem*{definition*}{Definition}

\newtheorem{innercustomdefi}{Definition}
\newenvironment{defi}[1]
  {\renewcommand\theinnercustomdefi{#1}\innercustomdefi}
  {\endinnercustomdefi}

%\newtheorem{example}{Example}

%\newtheorem{xca}[theorem]{Exercise}

\newtheorem{innercustomex}{Example}
\newenvironment{ex}[1]
  {\renewcommand\theinnercustomex{#1}\innercustomex}
  {\endinnercustomex}

\theoremstyle{remark}


%\newenvironment{proposition}[2][Proposition]{\begin{trivlist}
%\item[\hskip \labelsep {\bfseries #1}\hskip \labelsep {\bfseries #2.}]}{\end{trivlist}}


\newcommand{\abs}[1]{\left\lvert#1\right\rvert}

\definecolor{hotpink}{rgb}{1.0, 0, 0.6}
\definecolor{inter}{rgb}{0.4, 0.4, 1}
\definecolor{awesome}{rgb}{1.0, 0.13, 0.32}
\definecolor{darkblue}{rgb}{0.0, 0.0, 0.60}
\definecolor{darkgreen}{rgb}{0.0, 0.50, 0.0}
\definecolor{darkyellow}{rgb}{0.80, 0.80, 0.0}
\newcommand{\red}[1]{\textcolor{red}{#1}}
\newcommand{\rred}[1]{\textcolor{RubineRed}{#1}}
\newcommand{\dblue}[1]{\textcolor{darkblue}{#1}}
\newcommand{\dgreen}[1]{\textcolor{darkgreen}{#1}}
\newcommand{\blue}[1]{\textcolor{blue}{#1}}
\newcommand{\cyan}[1]{\textcolor{cyan}{#1}}
\newcommand{\magenta}[1]{\textcolor{magenta}{#1}}
\newcommand{\brown}[1]{\textcolor{brown}{#1}}
\newcommand{\dyellow}[1]{\textcolor{darkyellow}{#1}}
\newcommand{\am}[1]{\textcolor{Aquamarine}{#1}}

\newcommand{\purple}[1]{\textcolor{purple}{#1}}

\newcommand{\green}[1]{\textcolor{green}{#1}}
\newcommand{\orange}[1]{\textcolor{orange}{#1}}

\newcommand{\ot}{\otimes}
\newcommand{\rh}{\rho}
\newcommand{\tht}{\theta}
\usepackage{answers}
\usepackage{setspace}
\usepackage{graphicx}
\usepackage{enumerate}
\usepackage{multicol}
\usepackage{amssymb}
\usepackage{mathrsfs}
\usepackage[margin=1in]{geometry} 

\usepackage{braket}
\usepackage{CJKutf8}
\newcommand{\ob}[1]{\mkern 1.5mu\overline{\mkern-1.5mu#1\mkern-1.5mu}\mkern 1.5mu}
 
\newcommand{\N}{\mathbb{N}}
\newcommand{\iZ}{\mathbb{Z}}
\newcommand{\C}{\mathbb{C}}
\newcommand{\R}{\mathbb{R}}
\newcommand{\Q}{\mathbb{Q}}
\newcommand{\X}{\mathcal{X}}
\newcommand{\Y}{\mathcal{Y}}
\newcommand{\Z}{\mathbb{Z}}
\newcommand{\W}{\mathcal{W}}

\newcommand{\norm}[1]{\left\lVert#1\right\rVert}
\newcommand{\xra}[1]{\xrightarrow{#1}}
\newcommand{\E}{\mathbb{E}}
\newcommand{\V}{\operatorname{Var}} 
\newcommand {\tr} {\operatorname{Tr}}
\newcommand{\trp}[1]{\tr\left(#1\right)}
\newcommand{\trpp}[2]{\tr_{#1}\left(#2\right)}
\newcommand{\trb}[1]{\tr\left[#1\right]}

\newcommand*{\myfont}{\fontfamily{lmss}\selectfont}
\DeclareTextFontCommand{\fo}{\myfont}

%\usepackage{titlesec}

\makeatletter
%default definition of article.cls
%using \renewcommand instead of \newcommand
\renewcommand\part{%
   \if@noskipsec \leavevmode \fi
   \par
   \addvspace{4ex}%
   \@afterindentfalse
   \secdef\@part\@spart}

\def\@part[#1]#2{%
    \ifnum \c@secnumdepth >\m@ne
      \refstepcounter{part}%
      \addcontentsline{toc}{part}{\thepart\hspace{1em}#1}%
    \else
      \addcontentsline{toc}{part}{#1}%
    \fi
    {\parindent \z@ \raggedright
     \interlinepenalty \@M
     \normalfont
     \ifnum \c@secnumdepth >\m@ne
       \Large\bfseries \partname\nobreakspace\thepart
       \par\nobreak
     \fi
     \huge \bfseries #2%
     %%%\markboth{}{}\par}% removing redefinition of headings
     \par}%
    \nobreak
    \vskip 3ex
    \@afterheading}
\def\@spart#1{%
    {\parindent \z@ \raggedright
     \interlinepenalty \@M
     \normalfont
     \huge \bfseries #1\par}%
     \nobreak
     \vskip 3ex
     \@afterheading}
\makeatother

%\renewcommand{\partname}{Chapter}
%\newcommand{\part}{\part}
%\begin{comment}
%\titleformat{\part}
 % {\normalfont\myfont\huge\bfseries}
  %{\parttitlename\ \thepart}{20pt}{\huge}
%\end{comment}
  
  
\usepackage{etoolbox}
\patchcmd{\section}{\scshape}{\bfseries}{}{}
\makeatletter
\renewcommand{\@secnumfont}{\bfseries}
\makeatother  
  
%\renewcommand{\thesection}{\arabic{section}}  



\makeatletter
\def\@seccntformat#1{%
  \expandafter\ifx\csname c@#1\endcsname\c@section
  \thesection\hspace{2ex}
  \else
  \csname the#1\endcsname\quad
  \fi}
\makeatother

  
\makeatletter  
\renewcommand\partname{Chapter}
\renewcommand\part{\@startsection{part}{0}%
\z@{\linespacing\@plus\linespacing}{1\linespacing}%
{\myfont\bfseries\huge\raggedright}}  
\makeatother  
  
  

% http://joshua.smcvt.edu/latex2e/bs-at-startsection.html  
  
\makeatletter
\renewcommand\section{\@startsection{section}{1}%
{0pt}{.8\linespacing\@plus\linespacing}{.6\linespacing}%
{\LARGE\bfseries\color{black}}}
\makeatother

\makeatletter
\renewcommand\specialsection{\@startsection{specialsection}{1}%
{0pt}{.8\linespacing\@plus\linespacing}{.6\linespacing}%
{\LARGE\bfseries\color{hotpink}}}
\makeatother

\makeatletter
\renewcommand\subsection{\@startsection{subsection}{2}%
{0pt}{.8\linespacing\@plus.9\linespacing}{.7\linespacing}%
{\Large\bfseries\color{black}}}
\makeatother

\makeatletter
\renewcommand\subsubsection{\@startsection{subsubsection}{3}%
\z@{.6\linespacing\@plus.7\linespacing}{.4\linespacing}%
{\large\bfseries\color{black}}}
\makeatother

%https://tex.stackexchange.com/questions/268706/amsart-change-subsection-headings-from-boldface-to-smallcaps
%https://tex.stackexchange.com/questions/60437/newlines-after-section-headings-in-amsart
%http://ftp.ktug.org/tex-archive/macros/latex/required/amscls/doc/amsclass.pdf 의 Line 1175

%Since amsart.cls has
%\def\subsection{\@startsection{subsection}{2}%
%  \z@{.5\linespacing\@plus.7\linespacing}{-.5em}%
%  {\normalfont\bfseries}}
%you just add, in your document preamble (that is, before \begin{document}),
%\makeatletter
%\renewcommand\subsection{\@startsection{subsection}{2}%
%  \z@{.5\linespacing\@plus.7\linespacing}{-.5em}%
%  {\normalfont\scshape}}
%\makeatother

\newcommand{\subsec}[1]{\begin{adjustwidth}{-0.1in}{0in}\subsection{#1}\end{adjustwidth}}
\newcommand{\subsect}[1]{\hspace{-0.2in}\subsection{#1}}
\newcommand{\secc}[1]{\begin{adjustwidth}{-0.1in}{0in}\section{#1}\end{adjustwidth}}

\usepackage{array}
\preto\tabular{\setcounter{magicrownumbers}{0}}
\newcounter{magicrownumbers}
\newcommand\rownumber{\small\stepcounter{magicrownumbers}\arabic{magicrownumbers}}


\newcommand{\vb}{\vec{B}}
\newcommand{\vj}{\vec{j}}
\newcommand{\vn}{\vec{\nabla}}
\newcommand{\vr}{\mathbf{r}}
\newcommand{\vk}{\mathbf{k}}
\newcommand{\va}{\vec{A}}
\newcommand{\vl}{\vect{l}}
\newcommand{\vp}{\varphi}\newcommand{\hvp}{\hat{\varphi}}
\setcounter{section}{+0}

\newcommand{\hq}{\hat{Q}}
\newcommand{\hp}{\hat{\Phi}}

\newcommand{\ha}{\hat{a}}
\newcommand{\hN}{\hat{N}}
\newcommand{\ld}{\lambda}
\newcommand{\htp}{\hat{p}}

\newcommand{\om}{\omega}
\newcommand{\var}{\operatorname{Var}}
\newcommand{\vect}{\mathbf}
\newcommand{\nul}{\operatorname{Nul}}
\newcommand{\col}{\operatorname{Col}}
\newcommand{\row}{\operatorname{Row}}
\newcommand{\dg}{\dagger}
%    Blank box placeholder for figures (to avoid requiring any
%    particular graphics capabilities for printing this document).
\newcommand{\blankbox}[2]{%
  \parbox{\columnwidth}{\centering
%    Set fboxsep to 0 so that the actual size of the box will match the
%    given measurements more closely.
    \setlength{\fboxsep}{0pt}%
    \fbox{\raisebox{0pt}[#2]{\hspace{#1}}}%
  }%
}



\usepackage{bbm}
%     If your article inludes graphics, uncomment this command
\usepackage{svg}
\usepackage[super, square]{natbib}
%\usepackage[square]{natbib}

%\usepackage{biblatex}
%\bibliography{name.bib}
\svgsetup{inkscapelatex=false}
\usepackage{epstopdf}
\usepackage{url}
%\usepackage{circuitikz}

\usepackage{pgf}
\usepackage{graphicx}
\usepackage{pgfplots}

\usepackage{gnuplottex}
\usepackage{pgfplotstable}
\usepgfplotslibrary{external} 
\usetikzlibrary{external}
%\tikzexternalize[prefix=tikz/]


%\pgfplotsset{every axis/.append style={
                    %axis x line=middle,    % put the x axis in the middle
                    %axis y line=middle,    % put the y axis in the middle
                    %axis line style={<->,color=blue}, % arrows on the axis
                    %xlabel={$x$},          % default put x on x-axis
                    %ylabel={$y$},          % default put y on y-axis
            %}}
\pgfplotsset{every tick label/.append style={font=\tiny}}
\pgfplotsset{every x tick label/.append style={font=\tiny, yshift=0.5ex}}
\pgfplotsset{every y tick label/.append style={font=\tiny, xshift=0.5ex}}
\pgfplotsset{ /pgf/number format/precision=10}

\begin{comment}

\pgfplotsset{
    node near coord/.style={ % Style for activating the label for a single coordinate
        nodes near coords*={
            \ifnum\coordindex=#1\pgfmathprintnumber{\pgfplotspointmeta}\fi
        }
    },
    nodes near some coords/.style={ % Style for activating the label for a list of coordinates
        scatter/@pre marker code/.code={},% Reset the default scatter style, so we don't get coloured markers
        scatter/@post marker code/.code={},% 
        node near coord/.list={#1} % Run "node near coord" once for every element in the list
    }
}

\pgfplotsset{
    node near coord/.style={ % Style for activating the label for a single coordinate
        nodes near coords*={
            \ifnum\coordindex=#1\pgfmathprintnumber{\pgfplotspointmeta}\fi
        }
    },
    nodes near some coords/.style={ % Style for activating the label for a list of coordinates
        scatter/@pre marker code/.code={},% Reset the default scatter style, so we don't get coloured markers
        scatter/@post marker code/.code={},% 
        node near coord/.list={#1} % Run "node near coord" once for every element in the list
    }
}

\pgfplotsset{
    node near coord/.style args={#1/#2/#3}{% Style for activating the label for a single coordinate
        nodes near coords*={
            \ifnum\coordindex=#1 #2\fi
        },
        scatter/@pre marker code/.append code={
            \ifnum\coordindex=#1 \pgfplotsset{every node near coord/.append style=#3}\fi
        }
    },
    nodes near some coords/.style={ % Style for activating the label for a list of coordinates
        scatter/@pre marker code/.code={},% Reset the default scatter style, so we don't get coloured markers
        scatter/@post marker code/.code={},% 
        node near coord/.list={#1} % Run "node near coord" once for every element in the list
    }

}

\pgfplotsset{
    node near coord/.style args={#1/#2/#3}{% Style for activating the label for a single coordinate
        nodes near coords*={
            \ifnum\coordindex=#1  #2\fi
        },
        scatter/@pre marker code/.append code={
            \ifnum\coordindex=#1 \pgfplotsset{every node near coord/.append style=#3}\fi
        }
    },
    nodes near some coords/.style={ % Style for activating the label for a list of coordinates
        scatter/@pre marker code/.code={},% Reset the default scatter style, so we don't get coloured markers
        scatter/@post marker code/.code={},% 
        node near coord/.list={#1} % Run "node near coord" once for every element in the list
    }

}

\end{comment}





\pgfplotsset{
    node near coord/.style args={#1/#2/#3}{% Style for activating the label for a single coordinate
        nodes near coords*={
            \ifnum\coordindex=#1\pgfmathprintnumber{\pgfplotspointmeta}#2\fi
        },
        scatter/@pre marker code/.append code={
            \ifnum\coordindex=#1 \pgfplotsset{every node near coord/.append style=#3}\fi
        }
    },
    nodes near some coords/.style={ % Style for activating the label for a list of coordinates
        scatter/@pre marker code/.code={},% Reset the default scatter style, so we don't get coloured markers
        scatter/@post marker code/.code={},% 
        node near coord/.list={#1} % Run "node near coord" once for every element in the list
    }
}


\pgfplotsset{
    pin near coord/.style args={#1/#2/#3}{% Style for activating the label for a single coordinate
        scatter/@pre marker code/.append code={
           \ifnum 1=#3
            \ifnum\coordindex=#1  point meta=x, \node[pin={#2:\small\color{black} \SI{\small\pgfmathprintnumber{\pgfplotspointmeta}}{\volt}  }]{};\fi 
            \fi 
            \ifnum 2=#3
            \ifnum\coordindex=#1 point meta=x, \node[pin={[align=center]75:#2}]{};\fi
            \fi
        }
    },
    pins near some coords/.style={ % Style for activating the label for a list of coordinates
        scatter,
        scatter/@pre marker code/.code={},% Reset the default scatter style, so we don't get coloured markers
        scatter/@post marker code/.code={},% 
        pin near coord/.list={#1} % Run "pin near coord" once for every element in the list
    }
}
\pgfplotsset{select coords between index/.style 2 args={
    x filter/.code={
        \ifnum\coordindex<#1\def\pgfmathresult{}\fi
        \ifnum\coordindex>#2\def\pgfmathresult{}\fi
    }
}}
%\usetikzlibrary {math, fpu}
%\pgfkeys{/pgf/fpu = true}
\usepackage{csvsimple}
\usepackage{ifpdf}
\usepackage{sidecap}
\usepackage{float}
\usepackage[labelformat=simple]{subcaption}
\renewcommand{\thesubfigure}{(\scshape\Alph{subfigure})}
\captionsetup[subfigure]{position=top}
\begin{comment}
\usepackage[labelformat=simple]{subcaption}
\renewcommand\thesubfigure{(\alph{subfigure})}
(Note: Since parens is the default label format you will get double parentheses in sub-captions if you don't specify a different label format, e.g., simple.)
\end{comment}
%\renewcommand\thesubfigure{(\alph{subfigure})}
\newcommand{\squeezeup}{\vspace{-5mm}}
\newcommand{\squeezeupp}{\vspace{-8.5mm}}
\newcommand{\squeezeuppp}{\vspace{-10mm}}
%\usepackage{pgfplots}
   % \usetikzlibrary{intersections}
    % use this `compat' level or higher so that TikZ coordinates don't have to be prefixed
    % with `axis cs:'
   %\pgfplotsset{width=15cm,compat=1.11}

\usepackage{siunitx}\sisetup{parse-numbers=false}
\sisetup{parse-numbers=false}
\DeclareSIUnit\torr{torr}
\DeclareSIUnit\gauss{G}
\usepackage{amsmath}
\usepackage{amsthm,amssymb}

\makeatletter
\def\convertto#1#2{\strip@pt\dimexpr #2*65536/\number\dimexpr 1#1}
\makeatother

\usepackage{mhchem}%
\usepackage{chemformula}
\let\ce\ch
\usepackage{longtable}
\usepackage{multirow, makecell}
\usepackage{multicol}
\begin{comment}

\makeatletter
\renewenvironment{thebibliography}[1]
     {\begin{multicols}{2}[\section*{\refname}]%
      \@mkboth{\MakeUppercase\refname}{\MakeUppercase\refname}%
      \list{\@biblabel{\@arabic\c@enumiv}}%
           {\settowidth\labelwidth{\@biblabel{#1}}%
            \leftmargin\labelwidth
            \advance\leftmargin\labelsep
            \@openbib@code
            \usecounter{enumiv}%
            \let\p@enumiv\@empty
            \renewcommand\theenumiv{\@arabic\c@enumiv}}%
      \sloppy
      \clubpenalty4000
      \@clubpenalty \clubpenalty
      \widowpenalty4000%
      \sfcode`\.\@m}
     {\def\@noitemerr
       {\@latex@warning{Empty `thebibliography' environment}}%
      \endlist\end{multicols}}
\makeatother
%\url{tex.stackexchange.com/questions/20758/bibliography-in-two-columns-section-title-in-one/20761}
\end{comment}
%\renewcommand{\bibpreamble}{\begin{multicols}{2}}
%\renewcommand{\bibpostamble}{\end{multicols}}
\usepackage{helvet}

\usepackage{blindtext}
\usepackage{mathtools}
%\usepackage[x11names]{xcolor}
%\newtagform{blue}{\color{blue}(}{)}
%\usepackage[dvipsnames]{xcolor}

\definecolor{awesome}{rgb}{1.0, 0.13, 0.32}
\definecolor{darkblue}{rgb}{0.0, 0.0, 0.60} 


\newcommand{\re}{\operatorname{Re}}
%\newcommand{\im}{\operatorname{Im}}
\newcommand{\im}{\implies}

%\newcommand{\redheart}{\textcolor{red}{$\varheartsuit$}}
%\newcommand{\heart}{\ensuremath\varheartsuit}



\usepackage[english]{babel}
\usepackage[utf8]{inputenc}
% use KoTeX package for Korean 
\usepackage{kotex}






%\numberwithin{equation}{section}

\usepackage{answers}
\usepackage{setspace}

\usepackage{enumerate}
\usepackage{enumitem}
\usepackage{multicol}
\usepackage{mathrsfs}
\usepackage[margin=1in]{geometry} 
\usepackage{changepage}
\usepackage{braket}
\usepackage{CJKutf8}


\DeclareTextFontCommand{\fo}{\myfont}

\newcommand*{\helve}{\fontfamily{phv}\selectfont}
\DeclareTextFontCommand{\fohe}{\helve}

\makeatletter
\newcommand*{\rom}[1]{\expandafter\@slowromancap\romannumeral #1@}
\makeatother


\begin{comment}
\newcommand{\vb}{\vec{B}}
\newcommand{\vj}{\vec{j}}
\newcommand{\vn}{\vec{\nabla}}
\newcommand{\vr}{\mathbf{r}}
\newcommand{\vk}{\mathbf{k}}
\newcommand{\va}{\vec{A}}
\newcommand{\vl}{\vect{l}}
\newcommand{\vp}{\varphi}\newcommand{\hvp}{\hat{\varphi}}
\setcounter{section}{+11}

\newcommand{\hq}{\hat{Q}}
\newcommand{\hp}{\hat{\Phi}}

\newcommand{\ha}{\hat{a}}
\newcommand{\hN}{\hat{N}}
\newcommand{\ld}{\lambda}
\newcommand{\htp}{\hat{p}}
%\usepackage[T1]{fontenc}
%\usepackage{bold-extra}
\newcommand{\var}{\operatorname{Var}}
\newcommand{\vect}{\mathbf}
\newcommand{\nul}{\operatorname{Nul}}
\newcommand{\col}{\operatorname{Col}}
\newcommand{\row}{\operatorname{Row}}
\newcommand{\dg}{\dagger}
\end{comment}
\usepackage{fancyhdr}
\usepackage{regexpatch}

\begin{comment}
%%% let's patch the amsart macros
\makeatletter
\renewcommand{\sectionmark}[2]{%
  \ifnum#1<\@m
    \markboth{\thesection. #2}{\thesection. #2}%
  \else
    \markboth{#2}{#2}%
  \fi}
\xpatchcmd*{\@sect}
  {\@tocwrite}
  {\csname #1mark\endcsname{#2}{#7}\@tocwrite}
  {}{}
\makeatother
\end{comment}

\pagestyle{fancy}
\fancyhf{}
%\rhead{}
\lhead{ }
%\fancyhead[L]{\rightmark}
\fancyhead[R]{\textcolor{black}{\thepage}}

%\renewcommand{\subsectionmark}[1]{%
%  \markright{\MakeUppercase{\thesubsection.\ #1}}}%

%\renewcommand{\sectionmark}[1]{%
%\markboth{\thesection\quad #1}{}}
%\fancyhead{}
%\fancyfoot[L]{\leftmark}
%\fancyfoot{}
%\fancyfoot[CE,CO]{\leftmark}
\cfoot{\thepage}
\newlength\FHoffset
\setlength\FHoffset{0.12in}

\addtolength\headwidth{2\FHoffset}

\fancyheadoffset{\FHoffset}

\usepackage{layouts}



\setcounter{tocdepth}{3}% to get subsubsections in toc

\let\oldtocsection=\tocsection

\let\oldtocsubsection=\tocsubsection

\let\oldtocsubsubsection=\tocsubsubsection

\renewcommand{\tocsection}[2]{\hspace{0em}\oldtocsection{#1}{#2}}
\renewcommand{\tocsubsection}[2]{\hspace{1em}\oldtocsubsection{#1}{#2}}
\renewcommand{\tocsubsubsection}[2]{\hspace{2em}\oldtocsubsubsection{#1}{#2}}
\renewcommand{\contentsname}{Table of Contents}

\begin{document}
%\pgfset{fpu = true}
\usetagform{blue} \reqnomode
\newgeometry{left=0.9in, right=0.9in, top=1in, bottom=1in}
%\newgeometry{left=0.7in, right=0.7in, top=1in, bottom=1in}
\fancypagestyle{plain}{
  \fancyhf{}
  \lhead{ }
  \chead{}
  \rhead{\textcolor{black}{\thepage}}
}

\begin{center}
\linebreakc
\linebreakc
\linebreakc
\huge
%\textbf{\fo{ EXP 1.}}

\textbf{Pharmacology Summary}

\textcolor{white}{bl}\\

\Large

Gyunghee Han

\end{center}

\textcolor{white}{bl}\\

\normalsize

\renewcommand{\contentsname}{\normalfont \bfseries \Large Table of Contents}
\tableofcontents



\section{into patient? (I)}
\section{into site of action? (II)}
\section{producing required effect? (III) }
\section{Pharmacodynamics (IV)}
\subsection{Receptors}
\begin{defi}{1}{Receptor}
\begin{enumerate}
    \item binding affinity
    \item physiological responses
    \item ligand  specificity \& selectivity
    \item regional distribution\item $R_c$ is saturable \&  reversible
    \item binding site
\end{enumerate}

\end{defi}
\subsection{Equations (Single Binding Site) }
$ \xrightleftharpoons[k_2]{k_1} \longleftrightarrow$
\begin{equation}
    [R]_\text{tot} \ = \ [R] + [AR]  \qquad \implies \qquad [R] = [R]_\text{tot} - [AR] \label{eq:one}
\end{equation}
\begin{equation*}
\begin{gathered}
    [A]+[R] \  \longleftrightarrow \  [AR]
\end{gathered}    
\end{equation*}
\begin{equation}
        K_d \ := \ \frac{[A]\times [R]}{[AR]} \label{eq:two}
\end{equation}
Putting \cref{eq:one} into \cref{eq:two}, we get
\begin{align}
    K_d \ &= \ \frac{[A] \times \lf([R]_\text{tot} - [AR] \rg)}{[AR]} \nonumber\\
    &= \ \frac{[A][R]_\text{tot}}{[AR]}
    \ - \ [A] \label{eq:three}
\end{align}
Solving \cref{eq:three} for $[AR]$, 
\begin{equation*}
        \frac{[A][R]_\text{tot}}{[AR]}
= K_d + [A]         
\end{equation*}
\begin{equation}
\implies [AR] = \frac{[A][R]_\text{tot}}{K_d+[A]} \label{eq:ar}
\end{equation}

\vspace{5pt}


\subsubsection{}

When $\displaystyle [AR]$ equals $\displaystyle \dfrac{1}{2}[R]_\text{tot}$, substituting $\displaystyle \dfrac{1}{2}[R]_\text{tot}$ for LHS in \cref{eq:ar},
\begin{align*}
    \dfrac{1}{2}[R]_\text{tot} \ = \ \frac{[A][R]_\text{tot}}{K_d+[A]} 
\end{align*}
Canceling $[R]_\text{tot}$ in both sides of above equation,
\begin{align*}
    \frac{1}{2} = \frac{[A]}{[A]+K_d} 
\end{align*}
Therefore, $\displaystyle K_d\  =\  [A]_\text{half saturated} = [A] \vert_{[AR] = \frac{1}{2} [R]\text{tot}}$

\vspace{5pt}

\subsubsection{}
Rearranging \cref{eq:ar},

\begin{align*}
    [AR] \ \times \ (K_d +[A] ) \ = \ [A] \ \times \ [R]_\text{tot} \\
    [AR]\cdot K_d
    \ = \ [A]\cdot[R]_\text{tot} - [A]\cdot[AR] \\
    [AR]\cdot K_d 
    \ = \ -[A]\cdot([AR] - [R]_\text{tot}) \\
    \frac{[AR]}{[A]}
    \ = \ - \frac{1}{K_d}\  \cdot\  ([AR] \ - \ [R]_\text{tot}) 
\end{align*}

\subsection{Equation (Multiple Binding Sites) (Hill)}
\begin{equation}
    [R]_\text{tot} \ = \ [R] + [AR]  \qquad \implies \qquad [R] = [R]_\text{tot} - [AR] \label{eq:onem}
\end{equation}
\begin{equation*}
\begin{gathered}
    [A] + [R] \  \longleftrightarrow \  [AR]
\end{gathered}    
\end{equation*}
\begin{equation}
        K_d \ := \ \frac{[A]^n\times [R]^n}{[AR]} \label{eq:twom}
\end{equation}
Putting \cref{eq:onem} into \cref{eq:twom}, we get
\begin{align}
    K_d \ &= \ \frac{[A]^n \times \lf([R]_\text{tot} - [AR] \rg)\red{^n}}{[AR]} \nonumber\\
    &= \ \frac{[A][R]_\text{tot}}{[AR]}
    \ - \ [A] \label{eq:three}
\end{align}
Solving \cref{eq:three} for $[AR]$, 
\begin{equation*}
        \frac{[A][R]_\text{tot}}{[AR]}
= K_d + [A]         
\end{equation*}
\begin{equation}
\implies [AR] = \frac{[A][R]_\text{tot}}{K_d+[A]} \label{eq:ar}
\end{equation}

\vspace{5pt}

\begin{align*}
    K_d \ = \ \frac{[A]^n \cdot [R]^n}{[AR]} 
\end{align*}
\begin{align*}
[AR] \ = \ \frac{[A]^n\cdot [R}{}    
\end{align*}

\subsection{Paton}

\begin{align*}
[A] +[R]
    \xrightleftharpoons[\ k_2\ ]{\ k_1:\text{ onset rate}\ }[AR]
\end{align*}
\begin{align*}
    k_1\cdot[A]\cdot[R]
    \ = \ k_2\cdot[AR]
\end{align*}
\begin{align*}
\text{association rate }\ V_f \ := \ \frac{k_1 \cdot[A]\cdot[R]}{[R]_\text{tot}}     
\end{align*}
\subsection{Dose-Response Relationship}
\subsubsection{Quantal}
\begin{align*}
T_i \ = \ \frac{LD_{50}}{ED_{50}} \ \text{ or } \ \frac{TD_{50}}{ED_{50}}    
\end{align*}
\subsubsection{Graded}
\subsection{Drug  Interactions}
\subsubsection{Drug Drug Interactions}
\section{Therapeutic effect : appropriate therapeutic effect? (V)}






\section{0427 1}

\begin{equation}
    I \  = \ G_K \times (V_m - E_K)
\end{equation}
$G_K$ : 얼마나 많이 열려있는지 

activation curve : $G_K$ - membrane potential curve 

막전압-gated curve

activation될 확률 ($G_K$)가 전압이 depol되면 될수록 커진다. 

hyperpol쪽에서는 0,

depol쪽에서는 1이되는 

S자 모양의 함수이다. 


voltage-clamp 실험 : voltage에 따라 얼마나 activation 되는지 보는 실험 


2번은 관심있으면 보면됨 

3번은 잘 알아야함 


\end{document}
